% Quantum Data Structures: Towards Quantum-Accelerated Probabilistic Membership, Similarity, and Substring Search
\documentclass[11pt]{article}
\usepackage[utf8]{inputenc}
\usepackage{amsmath, amssymb, amsthm}
\usepackage{graphicx}
\usepackage{hyperref}
\usepackage{authblk}
\usepackage{enumitem}
\usepackage[numbers,sort&compress]{natbib}
\usepackage{geometry}
\geometry{margin=1in}


% Conference-ready title and author block
	itle{Quantum Data Structures: Quantum-Accelerated Probabilistic Membership, Similarity, and Substring Search}
\author[1]{Konstantin Krasovitskiy}
\affil[1]{Department of Computer Science, University of Cyprus \\ MSc in Artificial Intelligence}
\date{\today}

\begin{document}
\maketitle

\begin{abstract}
Probabilistic data structures such as Bloom filters, SimHash, and suffix arrays are essential for scalable search and similarity in large-scale systems. We ask: can quantum algorithms offer new trade-offs in accuracy, memory, and query performance for these primitives? We introduce and analyze three quantum data structures (QDS): Quantum Approximate Membership (QAM), Quantum Suffix Sketch (Q-SubSketch), and Quantum Similarity Hash (Q-SimHash). We provide quantum circuit constructions, theoretical bounds, and rigorous comparisons to classical baselines. Our results establish a foundation for quantum-accelerated data structures and motivate further exploration in quantum information processing.
\end{abstract}


\section{Introduction}
Probabilistic data structures are essential for scalable search, deduplication, and similarity in large-scale systems. Quantum computing enables fundamentally new algorithmic paradigms. This paper explores quantum analogues of classical data structures, focusing on their theoretical properties, quantum circuit constructions, and provable trade-offs.


\section{Related Work}
\begin{itemize}[leftmargin=*,itemsep=0.5em]
    \item \textbf{Classical probabilistic data structures:} Bloom filters~\cite{bloom1970space}, counting Bloom filters, SimHash~\cite{charikar2002similarity}, and suffix arrays~\cite{manber1993suffix} are widely used for approximate membership, similarity, and substring search. Recent high-performance baselines include Cuckoo filters, XOR filters, and Vacuum filters, which are included in our experimental comparisons.
    \item \textbf{Quantum search and hashing:} Grover's algorithm~\cite{grover1996fast} provides quadratic speedup for unstructured search. Quantum fingerprinting~\cite{buhrman2001quantum} and quantum Bloom filters~\cite{giovannetti2008quantum} have been proposed for efficient data representation and comparison.
    \item \textbf{Quantum circuit model:} We adopt the unitary circuit model with noisy measurements, as implemented in Qiskit~\cite{qiskit}.
\end{itemize}


\section{Computational Model and Evaluation Metrics}
We use the unitary circuit model with the following constraints:
\begin{itemize}[leftmargin=*,itemsep=0.5em]
    \item \textbf{Query/Update:} Unitary circuits with measurement at query time.
    \item \textbf{No-cloning:} Quantum mechanical no-cloning is respected.
    \item \textbf{Error model:} Depolarizing or Pauli noise per 2-qubit gate; measurement error $p_r$.
    \item \textbf{Cost metrics:} Time (gate depth), space (logical qubits), accuracy (false positive $\alpha$, false negative $\beta$), shot budget.
\end{itemize}
We compare QDS to classical baselines under matched memory and accuracy constraints.


\section{Classical Baselines}
We compare all quantum data structures to the following classical baselines:
\begin{itemize}[leftmargin=*,itemsep=0.5em]
    \item \textbf{Bloom filter:} Standard $k$-hash bit array.
    \item \textbf{Cuckoo filter:} Bucketed hash table with kick-out insertion, supporting deletions.
    \item \textbf{XOR filter:} Space-efficient, fast membership filter using XOR-based hashing.
    \item \textbf{Vacuum filter:} High-performance Bloom variant with improved false positive rates.
\end{itemize}
These baselines are implemented and evaluated alongside quantum approaches in all plots and tables.

\section{Quantum Data Structure Algorithms}
\subsection{Quantum Approximate Membership (QAM)}
QAM encodes set membership via phase rotations on a register of $m$ qubits, using $k$ hash functions. Querying is performed by re-applying the hash pattern and measuring overlap with the reference state. See Figure~\ref{fig:qam_circuit}.

\begin{figure}[h]
\centering
\includegraphics[width=0.7\textwidth]{qam_circuit.png}
\caption{QAM circuit diagram.}
\label{fig:qam_circuit}
\end{figure}

\subsection{Quantum Suffix Sketch (Q-SubSketch)}
Q-SubSketch encodes substrings of a text into a quantum register using phase encoding and stride-based hashing. Substring queries are performed by interference and measurement. This construction is related to classical suffix arrays and sketches.

\subsection{Quantum Similarity Hash (Q-SimHash)}
Q-SimHash encodes binary vectors into quantum states using $k$ phase rotations. Similarity queries are performed by measuring the overlap between two encoded states. This is related to classical SimHash and quantum fingerprinting.




\section{Theoretical Results}
We provide new analytical bounds, lower bounds, and a formal model for QAM:
\begin{itemize}[leftmargin=*,itemsep=0.5em]
    \item \textbf{Lemma 2.1 (Ideal QAM False-Positive Bound):} $\alpha \leq \exp(-C k (1-\rho))$ for load factor $\rho = |S|/m$. (See Appendix and `theory/qam_bounds.tex`)
    \item \textbf{Lemma 2.2 (Noise-Perturbed QAM Bound):} Under Pauli noise, the acceptance gap degrades $\leq O(k\epsilon)$. (See Appendix and `theory/qam_bounds.tex`)
    \item \textbf{QAM Lower Bound:} Any QAM scheme using $k$ hash families and $m$ qubits requires $\Omega(\log m)$ qubits to preserve distinguishability under Pauli noise. (See Appendix and `theory/qam_lower_bound.tex`)
    \item \textbf{Quantum Cell Probe Model:} We define a quantum analogue of the classical cell probe model, where memory is a qubit array, queries are phase-based fingerprints, and measurements return acceptance bits. (See Appendix and `theory/cell_probe_model.md`)
    \item \textbf{Batching:} Variance reduction factor $\sim B$ for batch size $B$.
    \item \textbf{QAM Deletion Limitation:} Deletion by inverse phase (Rz(-\theta)) does not reliably suppress the acceptance probability for deleted items due to quantum interference and hash collisions. See empirical results and discussion in Appendix and `theory/qam_deletion_limitations.md`.
\end{itemize}
Proofs, definitions, and details are provided in the Appendix and the `theory/` directory.


\section{Limitations and Future Work}
Current quantum hardware is limited in qubit count and noise. Simulation is memory-intensive. QAM deletion by inverse phase is not robust in practice (see Section~\ref{Theoretical Results} and Appendix). Open questions include error mitigation, dynamic/streaming QDS, robust quantum deletion, and hardware experiments.


\section*{Contributions}
This work was solely conducted by Konstantin Krasovitskiy, who designed, implemented, and evaluated all quantum and classical data structure baselines, developed the theoretical analysis, and prepared the manuscript.

\section*{Acknowledgments}
We thank the Qiskit community and prior authors for foundational work.

% --- Submission Checklist ---
% [ ] Title and abstract are concise and informative
% [ ] Author and institution information is correct
% [ ] All sections required by the conference CFP are present
% [ ] Citations use \cite and are valid
% [ ] Figures are referenced and included as needed
% [ ] Theoretical results are clearly stated and proved (see Appendix)
% [ ] No experimental results included (per user request)
% [ ] PDF compiles without errors or warnings
% [ ] Paper length and formatting comply with conference requirements


\bibliographystyle{plainnat}
\begin{thebibliography}{99}
\bibitem{bloom1970space} B. H. Bloom. Space/time trade-offs in hash coding with allowable errors. \emph{Communications of the ACM}, 1970.
\bibitem{charikar2002similarity} M. S. Charikar. Similarity estimation techniques from rounding algorithms. In \emph{STOC}, 2002.
\bibitem{manber1993suffix} U. Manber and G. Myers. Suffix arrays: A new method for on-line string searches. \emph{SIAM Journal on Computing}, 1993.
\bibitem{grover1996fast} L. K. Grover. A fast quantum mechanical algorithm for database search. In \emph{STOC}, 1996.
\bibitem{buhrman2001quantum} H. Buhrman, R. Cleve, J. Watrous, and R. de Wolf. Quantum fingerprinting. \emph{Physical Review Letters}, 2001.
\bibitem{giovannetti2008quantum} V. Giovannetti, S. Lloyd, and L. Maccone. Quantum random access memory. \emph{Physical Review Letters}, 2008.
\bibitem{qiskit} Qiskit: An open-source framework for quantum computing. \url{https://qiskit.org/}
\end{thebibliography}

\appendix
\section{Proofs and Additional Figures}
See the \texttt{theory/} directory for detailed proofs and additional plots.

\end{document}
